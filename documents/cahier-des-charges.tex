\documentclass{article}

\title{Cahier des charges du projet 3 en LINFO1001}

\author{
	Garzonio, Mathieu
	\and
	Rota-Benabid, Lucas
	\and
	Wibo, Aymeric
}

\begin{document}

\maketitle

\section{Objectif}

L'objectif de cette mission est la création d'un assistant principalement vocal pour les personnes d'un certain âge. Pour être plus spécifique, il nous est demandé deux fonctionnalités principales :

\begin{enumerate}

	\item{La possibilité de créer, modifier, et supprimer une liste de courses vocalement ;}
	\item{La possibilité d'enregistrer un code de carte bancaire sécurisé au moyen d'un mot de passe, et de pouvoir la rappeler.}

\end{enumerate}

\section{Liste des tâches}

\begin{itemize}
    \item {\underline{Tâches principales}}
    \begin{enumerate}
        \item {Listes de courses}
        \begin{itemize}
            \item {Créer, modifier, supprimer la liste + énonciation de la liste avec text-to-speech}
        \end{itemize}
        \item {Code de CB}
        \begin{itemize}
            \item {Enregistrer un code de CB, le sécuriser avec un mot de passe + énonciation du code avec text-to-speech/afficher à l'écran}
        \end{itemize}
    \end{enumerate}
    \item {\underline{Tâches supplémentaires} (en constante évolution)}
    \begin{enumerate}
        \item Briefing météo du jour via text-to-speech
        \item {Détection de chute via l'accéléromètre, compteur de +/- 10 secondes qui se déclenche, et qui, si aucune action pour l'arrêter (donc potentiellement une vraie chute) est prise, alerte tout appareil bluetooth à portée du ElderPi + alarme ? }
    \end{enumerate}
\end{itemize}
Le tout, de manière \underline{vocal} (ou quasi tout).
\end{document}
